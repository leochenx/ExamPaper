\documentclass[12pt,twoside]{article}
\usepackage{amsfonts}
\usepackage{mathbbold}
\usepackage{amsmath}
\usepackage{CJK}
\usepackage{CJKulem}
\usepackage{color}
\usepackage{colortbl}
\usepackage{fancyhdr}
\usepackage{graphicx}
\usepackage{indentfirst}
\usepackage{lastpage}
\usepackage{latexsym}
\usepackage{listings}
\usepackage{multirow}
%\usepackage{times} %use the Times New Roman fonts
\usepackage{mathrsfs}
\usepackage{placeins}
\usepackage{titlesec}
\usepackage{ulem}
\usepackage[paperwidth=19.5cm,paperheight=27cm,top=1.5cm,bottom=2cm,right=2cm]{geometry}
%\lstset{language=C,keywordstyle=\color{red},showstringspaces=false,rulesepcolor=\color{green}}

\oddsidemargin=0.5cm   %奇数页页边距
\evensidemargin=-0.6cm %偶数页页边距
\textwidth=15cm        %文本的宽度

%%%%%%%%%%%%%%%%%%%%%%%%%%%%%%%%%%%%%%%%%%%%%%%
%定义公式: \[ \] 靠左 其他居中

\newdimen\mathindent
\mathindent\leftmargini

\makeatletter

\renewcommand\[{\relax
               \ifmmode\@badmath
               \else
                 \begin{trivlist}%
                   \@beginparpenalty\predisplaypenalty
                   \@endparpenalty\postdisplaypenalty
                   \item[]\leavevmode
                   \hb@xt@\linewidth\bgroup $\m@th\displaystyle %$
                     \hskip\mathindent\bgroup
               \fi}
\renewcommand\]{\relax
               \ifmmode
                     \egroup $\hfil% $
                   \egroup
                 \end{trivlist}%
               \else \@badmath
               \fi}

\makeatother

%%%%%%%%%%%%%%%%%%%%%%%%%%%%%%%%%%%%%%%%%%%%%%%

%定义“命令”\kcmc (课程名称),\kcdm (课程代码), \ksxs (考试形式)
\newcommand{\kcmc}[1]{\settowidth{\labelwidth}{#1}
  \underline{\,\hspace{2.6cm}\hspace{-0.5\labelwidth}\mbox{\textbf{#1}}\hspace{2.6cm}\hspace{-0.5\labelwidth}}
}
\newcommand{\kcdm}[1]{\settowidth{\labelwidth}{#1}
  \underline{\,\hspace{1.5cm}\hspace{-0.5\labelwidth}\mbox{\textbf{#1}}\hspace{1.5cm}\hspace{-0.5\labelwidth}}
}
\newcommand{\ksxs}[1]{\settowidth{\labelwidth}{#1}
  \underline{\,\hspace{1.7cm}\hspace{-0.5\labelwidth}\mbox{\textbf{#1}}\hspace{1.7cm}\hspace{-0.5\labelwidth}}
}

%%%%%%%%%%%%%%%%%%%%%%%%%%%%%%%%%%%%%%%%%%%%%%%

%定义划线命令

\newcommand{\fixblank}{\uline{\hspace{1em}\ \hspace{1em}\
\hspace{1em}\ \hspace{1em}\ \hspace{1em}\ \hspace{1em}}}
\newcommand{\blank}[1]{\uline{{}\hspace{#1em}{}}}

%%%%%%%%%%%%%%%%%%%%%%%%%%%%%%%%%%%%%%%%%%%%%%%

\renewcommand{\baselinestretch}{1.5}

\renewcommand{\headrulewidth}{0pt}
\pagestyle{fancy}


\begin{document}
\begin{CJK}{GBK}{song}

%页码1
\fancyhf{} \fancyfoot[CO,CE]{\vspace*{1mm}第\,\thepage\,页 \, (\, 共
\pageref{LastPage}页\,)}

%%%%%%%%%%%%%%%%%%%%%%%%%%%%%%%%%%%%%%%%%%%%%%%
%定义装订线

\newsavebox{\zdx}

\newcommand{\putzdx}{\marginpar{
\parbox{1cm}{\vspace{-2.6cm}
\rotatebox[origin=c]{90}{
\usebox{\zdx}
}} }}

%%%%%%

\sbox{\zdx}
    {\parbox{27cm}{
    \centering \CJKfamily{song}{{}\vspace{13mm}(\  \rotatebox[origin=c]{270}{题}
    \rotatebox[origin=c]{270}{答}\
    \rotatebox[origin=c]{270}{要}\ \rotatebox[origin=c]{270}{不}
    \rotatebox[origin=c]{270}{内}\
    \rotatebox[origin=c]{270}{线}\ \rotatebox[origin=c]{270}{订}\ \rotatebox[origin=c]{270}{装}
\ )} \\
\vspace{2mm}
\hspace{-40mm}\dotfill{} \dotfill{}\dotfill{}\dotfill{}\hspace{-100mm}{} \\
}}

\reversemarginpar

%第一页
\putzdx %%装订线--奇页数

%% 试卷头

  \begin{center}
    \begin{Large}
       复旦大学数学科学学院\\
        2005~2006学年第二学期期末考试试卷\vspace{2mm}\\
        {} $\Box$  \textbf{A 卷} \hspace{1cm}$\Box$ \textbf{B 卷}
    \end{Large}
  \end{center}

\noindent\large
\textbf{课程名称}:\kcmc{数学分析II}\hspace{2mm}\textbf{课程代码}:\kcdm{318.151.2}\\
\textbf{开课院系}:\kcmc{数学科学学院}\hspace{2mm}\textbf{考试形式}:\ksxs{闭卷}\\
\textbf{姓\ \ 名}:\underline{\, \hspace{2.2cm}}\hspace{2mm}
\textbf{学\ \ 号}: \underline{\, \hspace{2.0cm}}\hspace{2mm}
\textbf{专\ \ 业}:\underline{\, \hspace{3.2cm}}
%在“数学分析II”的位置填入课程名称;在“318.151.2”的位置填入课程代码
%在“闭卷”的位置填入考试形式:开卷、闭卷或课程论文

\bigskip

\begin{table}[htbp]\centering
\begin{tabular}{|c|c|c|c|c|c|c|c|c|c|}
\hline
\textbf{题\ \ 目}& 1 & 2 & 3 & 4 & 5 & 6 & 7 & 8 & \textbf{总分}\\
\hline \textbf{得\ \ 分}& \hspace{10mm} & \hspace{10mm} &
\hspace{10mm} &\hspace{10mm}
& \hspace{10mm} & \hspace{10mm} & \hspace{10mm} & \hspace{10mm} & \hspace{10mm} \\
\hline
\end{tabular}
\end{table}

\bigskip

\bigskip
%如果题目多于八题

\noindent
%\begin{table}[htbp]\centering
\begin{tabular}{|c|c|c|c|c|c|c|c|c|c|}
\hline
\textbf{题\ \ 号}& 1 & 2 & 3 & 4 & 5 & 6 & 7 & 8 & \multirow{2}{10pt}{\textbf{总\ \ 分}}\\
\cline{1-9} \textbf{得\ \ 分}& \hspace{10mm} & \hspace{10mm} &
\hspace{10mm} &\hspace{10mm}
& \hspace{10mm} & \hspace{10mm} & \hspace{10mm} & \hspace{10mm} & \hspace{10mm} \\
\hline
\textbf{题\ \ 号} & 9 & 10 & 11 & 12 & 13 & 14 & 15 & 16 &  \multirow{2}{10pt}{} \\
\cline{1-9} \textbf{得\ \ 分} & \hspace{10mm} & \hspace{10mm} &
\hspace{10mm} &\hspace{10mm}
& \hspace{10mm} & \hspace{10mm} & \hspace{10mm} & \hspace{10mm}& \hspace{10mm}\\
\hline
\end{tabular}

\bigskip

\bigskip\noindent \textbf{填空题}\begin{enumerate}\item
函数~$y=\frac{1}{\sqrt{25-x^2}}+\arcsin\frac{x-1}{5}$~的定义域为$\uline{\hspace{1.5 cm}}$。
\item
已知~$f(x)=e^{x^2}, f[\varphi(x)]=1-x$
且~$\varphi(x)\geq 0$,则~$\varphi(x)=\uline{\hspace{1.5 cm}}$,
定义域为$\uline{\hspace{1.5 cm}}$。
\item
~$\lim\limits_{n\rightarrow \infty}{(\frac{n+1}{n})^{(-1)^n}}=\uline{\hspace{1.5 cm}}.$~
\item
~$\lim\limits_{x\rightarrow -\infty}{\frac{\sqrt{4x^2+x-1}+x+1}{\sqrt{x^2+\sin x}}}$.
\item
~$\lim\limits_{x\rightarrow 0}{(1+x)^{\frac{2}{x}}}=\uline{\hspace{1.5 cm}}$.
\end{enumerate}
\noindent \textbf{选择题}\begin{enumerate}\item
选择以下两题中给出的四个结论中一个正确的结论。\\
(1)设~$f(x)=2^x+3^x-2$,则当~$x\rightarrow 0$~时,有(\hspace{1.5 cm})。
\begin{flalign*}
&(A)\text{~f(x)~与~x~是等价无穷小。} & &(B)\text{~f(x)~与~x~同阶但非等价无穷小。} \\
&(C)\text{~f(x)~是比~x~高阶的无穷小。} & &(D)\text{~f(x)~是比~x~低价无穷小。}
\end{flalign*}

(2)设
$$f(x)=\frac{e^{\frac{1}{x}}-1}{e^{\frac{1}{x}}+1},$$
则~$x=0$~是~$f(x)$~的(\hspace{1.5 cm})。
\begin{flalign*}
&(A)\text{可去间断点。} & &(B)\text{跳跃间断点。} \\
&(C)\text{第二类间断点。} & &(D)\text{连续点。}
\end{flalign*}
\item
对任意给定的~$\epsilon\in (0, 1),$~总存在正整数~$N$,~当~$n\geq N$~时,恒有~$|x_n-a|\leq 2\epsilon$~是
数列~$\{a_n\}$~收敛于~$a$~的~$\uline{\hspace{1.5 cm}}.$
\begin{flalign*}
&~(A)~\text{充分条件但非必要条件}~& &~(B)~\text{必要条件但非充分条件}~\\
&~(C)~\text{充分必要条件}~& &~(D)~\text{既非充分条件又非必要条件}~\\
\end{flalign*}
\item
设~$\{a_n\}, \{b_n\}, \{c_n\}$~均为非负数列,且~$\lim\limits_{n\rightarrow \infty} {a_n}=0$~,
~$\lim\limits_{n\rightarrow \infty} {b_n}=1$~,~$\lim\limits_{n\rightarrow \infty} {c_n}=\infty$~,
则必有~$\uline{\hspace{1.5 cm}}.$
\begin{flalign*}
&~(A)~a_n<b_n~\text{对任意~n~成立}~& &~(B)~b_n<c_n~\text{对任意~n~成立}~\\
&~(C)~\text{极限}~\lim\limits_{n\rightarrow \infty} {a_n}~c_n~\text{不存在}& &
~(D)~\text{极限}~\lim\limits_{n\rightarrow \infty} {b_n}~c_n~\text{不存在}
\end{flalign*}
\item
若~$\lim\limits_{x\rightarrow 0}{\frac{\sin 6x+xf(x)}{x^3}}=0$,则~$\lim\limits_{x\rightarrow 0}{\frac{6+f(x)}{x^2}}$为~$\uline{\hspace{1.5 cm}}$~.
\item
已知~$\lim\limits_{x\rightarrow +\infty}{(\frac{x^2}{x+1}-ax-b)}=0$~,其中~$a, b$~是常数,则\\
$(A)a=1, b=1\quad (B)a=-1, b=1\quad (C)a=1, b=-1\quad (D)a=-1, b=-1$
\end{enumerate}
\noindent \textbf{判断题}\begin{enumerate}\item
设~$f(x)$~为奇函数,判断下列函数的奇偶性:\\
\begin{flalign*}
&~(1)~xf(x)~& &~(2)~(x^2+1)f(x)~& &~(3)~|f(x)|~\\
&~(4)~-f(-x)~& &~(5)~f(x)(\frac{2^x+1}{2}-\frac{1}{2})~& &\\
\end{flalign*}
\item
判断函数~$f(x)=\ln(x+\sqrt{1+x^2})$~的奇偶性。
\end{enumerate}
\noindent \textbf{计算题}\begin{enumerate}\item
设~$a_0=0, a_{n+1}=1+\sin(a_n-1) (n\geq 0).$~求~$\lim\limits_{n\rightarrow \infty}{a_n}$.
\item
求函数
$$f(x)=\frac{|x|}{x}, g(x)=\frac{1-a^{\frac{1}{x}}}{1+a^{\frac{1}{x}}}(a>1)$$
当~$x\rightarrow 0$~时的左、右极限,并说明~$x\rightarrow 0$~时极限是否存在。
\end{enumerate}
\noindent \textbf{证明题}\begin{enumerate}\item
证明
$$\lim\limits_{n\rightarrow\infty}{(\frac{1}{\sqrt{n^2+1}}+\frac{1}{\sqrt{n^2+2}}+\cdots+\frac{1}{\sqrt{n^2+n}})}=1.$$
\item
证明~$\lim\limits_{x\rightarrow +\infty}{x\sin x}$~不存在.
\end{enumerate}
\clearpage
\end{CJK}
\end{document}