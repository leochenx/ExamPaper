\documentclass[12pt,twoside]{article}
\usepackage{amsfonts}
\usepackage{mathbbold}
\usepackage{amsmath}
\usepackage{CJK}
\usepackage{CJKulem}
\usepackage{color}
\usepackage{colortbl}
\usepackage{fancyhdr}
\usepackage{graphicx}
\usepackage{indentfirst}
\usepackage{lastpage}
\usepackage{latexsym}
\usepackage{listings}
\usepackage{multirow}
%\usepackage{times} %use the Times New Roman fonts
\usepackage{mathrsfs}
\usepackage{placeins}
\usepackage{titlesec}
\usepackage{ulem}
\usepackage[paperwidth=19.5cm,paperheight=27cm,top=1.5cm,bottom=2cm,right=2cm]{geometry}
%\lstset{language=C,keywordstyle=\color{red},showstringspaces=false,rulesepcolor=\color{green}}

\oddsidemargin=0.5cm   %奇数页页边距
\evensidemargin=-0.6cm %偶数页页边距
\textwidth=15cm        %文本的宽度

%%%%%%%%%%%%%%%%%%%%%%%%%%%%%%%%%%%%%%%%%%%%%%%
%定义公式: \[ \] 靠左 其他居中

\newdimen\mathindent
\mathindent\leftmargini

\makeatletter

\renewcommand\[{\relax
               \ifmmode\@badmath
               \else
                 \begin{trivlist}%
                   \@beginparpenalty\predisplaypenalty
                   \@endparpenalty\postdisplaypenalty
                   \item[]\leavevmode
                   \hb@xt@\linewidth\bgroup $\m@th\displaystyle %$
                     \hskip\mathindent\bgroup
               \fi}
\renewcommand\]{\relax
               \ifmmode
                     \egroup $\hfil% $
                   \egroup
                 \end{trivlist}%
               \else \@badmath
               \fi}

\makeatother

%%%%%%%%%%%%%%%%%%%%%%%%%%%%%%%%%%%%%%%%%%%%%%%

%定义“命令”\kcmc (课程名称),\kcdm (课程代码), \ksxs (考试形式)
\newcommand{\kcmc}[1]{\settowidth{\labelwidth}{#1}
  \underline{\,\hspace{2.6cm}\hspace{-0.5\labelwidth}\mbox{\textbf{#1}}\hspace{2.6cm}\hspace{-0.5\labelwidth}}
}
\newcommand{\kcdm}[1]{\settowidth{\labelwidth}{#1}
  \underline{\,\hspace{1.5cm}\hspace{-0.5\labelwidth}\mbox{\textbf{#1}}\hspace{1.5cm}\hspace{-0.5\labelwidth}}
}
\newcommand{\ksxs}[1]{\settowidth{\labelwidth}{#1}
  \underline{\,\hspace{1.7cm}\hspace{-0.5\labelwidth}\mbox{\textbf{#1}}\hspace{1.7cm}\hspace{-0.5\labelwidth}}
}

%%%%%%%%%%%%%%%%%%%%%%%%%%%%%%%%%%%%%%%%%%%%%%%

%定义划线命令

\newcommand{\fixblank}{\uline{\hspace{1em}\ \hspace{1em}\
\hspace{1em}\ \hspace{1em}\ \hspace{1em}\ \hspace{1em}}}
\newcommand{\blank}[1]{\uline{{}\hspace{#1em}{}}}

%%%%%%%%%%%%%%%%%%%%%%%%%%%%%%%%%%%%%%%%%%%%%%%

\renewcommand{\baselinestretch}{1.5}

\renewcommand{\headrulewidth}{0pt}
\pagestyle{fancy}


\begin{document}
\begin{CJK}{GBK}{song}

%页码1
\fancyhf{} \fancyfoot[CO,CE]{\vspace*{1mm}第\,\thepage\,页 \, (\, 共
\pageref{LastPage}页\,)}

%%%%%%%%%%%%%%%%%%%%%%%%%%%%%%%%%%%%%%%%%%%%%%%
%定义装订线

\newsavebox{\zdx}

\newcommand{\putzdx}{\marginpar{
\parbox{1cm}{\vspace{-2.6cm}
\rotatebox[origin=c]{90}{
\usebox{\zdx}
}} }}

%%%%%%

\sbox{\zdx}
    {\parbox{27cm}{
    \centering \CJKfamily{song}{{}\vspace{13mm}(\  \rotatebox[origin=c]{270}{题}
    \rotatebox[origin=c]{270}{答}\
    \rotatebox[origin=c]{270}{要}\ \rotatebox[origin=c]{270}{不}
    \rotatebox[origin=c]{270}{内}\
    \rotatebox[origin=c]{270}{线}\ \rotatebox[origin=c]{270}{订}\ \rotatebox[origin=c]{270}{装}
\ )} \\
\vspace{2mm}
\hspace{-40mm}\dotfill{} \dotfill{}\dotfill{}\dotfill{}\hspace{-100mm}{} \\
}}

\reversemarginpar

%第一页
\putzdx %%装订线--奇页数

%% 试卷头

  \begin{center}
    \begin{Large}
       复旦大学数学科学学院\\
        2005~2006学年第二学期期末考试试卷\vspace{2mm}\\
        {} $\Box$  \textbf{A 卷} \hspace{1cm}$\Box$ \textbf{B 卷}
    \end{Large}
  \end{center}

\noindent\large
\textbf{课程名称}:\kcmc{数学分析II}\hspace{2mm}\textbf{课程代码}:\kcdm{318.151.2}\\
\textbf{开课院系}:\kcmc{数学科学学院}\hspace{2mm}\textbf{考试形式}:\ksxs{闭卷}\\
\textbf{姓\ \ 名}:\underline{\, \hspace{2.2cm}}\hspace{2mm}
\textbf{学\ \ 号}: \underline{\, \hspace{2.0cm}}\hspace{2mm}
\textbf{专\ \ 业}:\underline{\, \hspace{3.2cm}}
%在“数学分析II”的位置填入课程名称;在“318.151.2”的位置填入课程代码
%在“闭卷”的位置填入考试形式:开卷、闭卷或课程论文

\bigskip

\begin{table}[htbp]\centering
\begin{tabular}{|c|c|c|c|c|c|c|c|c|c|}
\hline
\textbf{题\ \ 目}& 1 & 2 & 3 & 4 & 5 & 6 & 7 & 8 & \textbf{总分}\\
\hline \textbf{得\ \ 分}& \hspace{10mm} & \hspace{10mm} &
\hspace{10mm} &\hspace{10mm}
& \hspace{10mm} & \hspace{10mm} & \hspace{10mm} & \hspace{10mm} & \hspace{10mm} \\
\hline
\end{tabular}
\end{table}

\bigskip

\bigskip
%如果题目多于八题

\noindent
%\begin{table}[htbp]\centering
\begin{tabular}{|c|c|c|c|c|c|c|c|c|c|}
\hline
\textbf{题\ \ 号}& 1 & 2 & 3 & 4 & 5 & 6 & 7 & 8 & \multirow{2}{10pt}{\textbf{总\ \ 分}}\\
\cline{1-9} \textbf{得\ \ 分}& \hspace{10mm} & \hspace{10mm} &
\hspace{10mm} &\hspace{10mm}
& \hspace{10mm} & \hspace{10mm} & \hspace{10mm} & \hspace{10mm} & \hspace{10mm} \\
\hline
\textbf{题\ \ 号} & 9 & 10 & 11 & 12 & 13 & 14 & 15 & 16 &  \multirow{2}{10pt}{} \\
\cline{1-9} \textbf{得\ \ 分} & \hspace{10mm} & \hspace{10mm} &
\hspace{10mm} &\hspace{10mm}
& \hspace{10mm} & \hspace{10mm} & \hspace{10mm} & \hspace{10mm}& \hspace{10mm}\\
\hline
\end{tabular}

\bigskip

\bigskip\noindent \textbf{填空题}\begin{enumerate}\item
函数~$y=\frac{1}{\sqrt{25-x^2}}+\arcsin\frac{x-1}{5}$~的定义域为$\uline{\hspace{1.5 cm}}$。
\item
已知~$f(x)=\sin x, f[\varphi(x)]=1-x^2$,则~$\varphi(x)$~的定义域为$\uline{\hspace{1.5 cm}}$。
\item
已知~$f(x)=e^{x^2}, f[\varphi(x)]=1-x$
且~$\varphi(x)\geq 0$,则~$\varphi(x)=\uline{\hspace{1.5 cm}}$,
定义域为$\uline{\hspace{1.5 cm}}$。
\item
已知函数~$f(\log_ax)=\sqrt{x}$,则~$f(x)=\uline{\hspace{1.5 cm}}$,其定义域为$\uline{\hspace{1.5 cm}}$,
其中~$a\neq 0$,且~$a>0$~。
\item
设~$f(x)=\tan x, f[g(x)]=x^2-2$,且~$|g(x)|\leq\frac{\pi}{4}$,则~$g(x)$~的定义域为$\uline{\hspace{1.5 cm}}$。
\end{enumerate}
\noindent \textbf{选择题}\begin{enumerate}\item
设对任意的~$x$~,总有~$\varphi(x)\leq f(x)\leq g(x),$~且~$\lim\limits_{x\rightarrow \infty}[g(x)-\varphi(x)]=0,$~
则~$\lim\limits_{x\rightarrow \infty} {f(x)}=\uline{\hspace{1.5 cm}}.$~\\
(A)~存在且等于零~\quad (B)~存在但不一定为零~\quad (C)~一定不存在~\quad (D)~不一定存在~
\item
函数~$f(x)=x\sin x\uline{\hspace{1.5 cm}}.$~
\begin{flalign*}
&~(A)~\text{当}~x\rightarrow \infty~\text{时为无穷大}~& &~(B)~\text{在}~(-\infty, +\infty)~\text{内有界}~\\
&~(C)~\text{在}~(-\infty, +\infty)~\text{内无界}~& &~(D)~\text{当}~x\rightarrow \infty~\text{时有有限极限}~
\end{flalign*}
\item
数列~$x_n$~与~$y_n$~满足~$\lim\limits_{n\rightarrow \infty}{x_ny_n}=0,$~则下列断言正确的是~$\uline{\hspace{1.5 cm}}.$~
\begin{flalign*}
&~(A)~\text{若}~x_n~\text{发散},\text{则}~y_n~\text{必发散}~& &~(B)~\text{若}~x_n~\text{有界},\text{则}~y_n~\text{必有界}~\\
&~(C)~\text{若}~x_n~\text{有界},\text{则}~y_n~\text{必为无穷小}~& &~(D)~\text{若}~\frac{1}{x_n}~\text{为无穷小},\text{则}~y_n~\text{必为无穷小}~
\end{flalign*}
\item
若~$\lim\limits_{x\rightarrow 0}{\frac{\sin 6x+xf(x)}{x^3}}=0$,则~$\lim\limits_{x\rightarrow 0}{\frac{6+f(x)}{x^2}}$为~$\uline{\hspace{1.5 cm}}$~.
\item
当~$x\rightarrow 0$~时,~$(1-\cos x)\ln(1+x^2)$~是比~$x\sin x^n$~高阶的无穷小,而~$x\sin x^n$~是比~$(e^{x^2}-1)$~高阶的无穷小,
则正整数~$n=\uline{\hspace{1.5 cm}}.$~\\
$(A)~1\quad (B)~2\quad (C)~3 \quad(D)~4$
\end{enumerate}
\noindent \textbf{计算题}\begin{enumerate}\item
求~$\lim\limits_{n\rightarrow \infty}{(\frac{1}{n^2+n+1}+\frac{2}{n^2+n+1}+\cdots+\frac{n}{n^2+n+1})}$~.
\item
设~$a_0=0, a_{n+1}=1+\sin(a_n-1) (n\geq 0).$~求~$\lim\limits_{n\rightarrow \infty}{a_n}$.
\end{enumerate}
\noindent \textbf{证明题}\begin{enumerate}\item
证明:数列~$x_n=(-1)^n\cdot\frac{n+1}{n}$~是发散的。
\item
证明数列~$\sqrt{2}, \sqrt{2+\sqrt{2}},\sqrt{2+\sqrt{2+\sqrt{2}}},\cdots$~的极限存在,并求该极限。
\end{enumerate}
\clearpage
\end{CJK}
\end{document}