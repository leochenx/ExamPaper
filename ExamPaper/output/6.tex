\documentclass[12pt,twoside]{article}
\usepackage{amsfonts}
\usepackage{mathbbold}
\usepackage{amsmath}
\usepackage{CJK}
\usepackage{CJKulem}
\usepackage{color}
\usepackage{colortbl}
\usepackage{fancyhdr}
\usepackage{graphicx}
\usepackage{indentfirst}
\usepackage{lastpage}
\usepackage{latexsym}
\usepackage{listings}
\usepackage{multirow}
%\usepackage{times} %use the Times New Roman fonts
\usepackage{mathrsfs}
\usepackage{placeins}
\usepackage{titlesec}
\usepackage{ulem}
\usepackage[paperwidth=19.5cm,paperheight=27cm,top=1.5cm,bottom=2cm,right=2cm]{geometry}
%\lstset{language=C,keywordstyle=\color{red},showstringspaces=false,rulesepcolor=\color{green}}

\oddsidemargin=0.5cm   %奇数页页边距
\evensidemargin=-0.6cm %偶数页页边距
\textwidth=15cm        %文本的宽度

%%%%%%%%%%%%%%%%%%%%%%%%%%%%%%%%%%%%%%%%%%%%%%%
%定义公式: \[ \] 靠左 其他居中

\newdimen\mathindent
\mathindent\leftmargini

\makeatletter

\renewcommand\[{\relax
               \ifmmode\@badmath
               \else
                 \begin{trivlist}%
                   \@beginparpenalty\predisplaypenalty
                   \@endparpenalty\postdisplaypenalty
                   \item[]\leavevmode
                   \hb@xt@\linewidth\bgroup $\m@th\displaystyle %$
                     \hskip\mathindent\bgroup
               \fi}
\renewcommand\]{\relax
               \ifmmode
                     \egroup $\hfil% $
                   \egroup
                 \end{trivlist}%
               \else \@badmath
               \fi}

\makeatother

%%%%%%%%%%%%%%%%%%%%%%%%%%%%%%%%%%%%%%%%%%%%%%%

%定义“命令”\kcmc (课程名称),\kcdm (课程代码), \ksxs (考试形式)
\newcommand{\kcmc}[1]{\settowidth{\labelwidth}{#1}
  \underline{\,\hspace{2.6cm}\hspace{-0.5\labelwidth}\mbox{\textbf{#1}}\hspace{2.6cm}\hspace{-0.5\labelwidth}}
}
\newcommand{\kcdm}[1]{\settowidth{\labelwidth}{#1}
  \underline{\,\hspace{1.5cm}\hspace{-0.5\labelwidth}\mbox{\textbf{#1}}\hspace{1.5cm}\hspace{-0.5\labelwidth}}
}
\newcommand{\ksxs}[1]{\settowidth{\labelwidth}{#1}
  \underline{\,\hspace{1.7cm}\hspace{-0.5\labelwidth}\mbox{\textbf{#1}}\hspace{1.7cm}\hspace{-0.5\labelwidth}}
}

%%%%%%%%%%%%%%%%%%%%%%%%%%%%%%%%%%%%%%%%%%%%%%%

%定义划线命令

\newcommand{\fixblank}{\uline{\hspace{1em}\ \hspace{1em}\
\hspace{1em}\ \hspace{1em}\ \hspace{1em}\ \hspace{1em}}}
\newcommand{\blank}[1]{\uline{{}\hspace{#1em}{}}}

%%%%%%%%%%%%%%%%%%%%%%%%%%%%%%%%%%%%%%%%%%%%%%%

\renewcommand{\baselinestretch}{1.5}

\renewcommand{\headrulewidth}{0pt}
\pagestyle{fancy}


\begin{document}
\begin{CJK}{GBK}{song}

%页码1
\fancyhf{} \fancyfoot[CO,CE]{\vspace*{1mm}第\,\thepage\,页 \, (\, 共
\pageref{LastPage}页\,)}

%%%%%%%%%%%%%%%%%%%%%%%%%%%%%%%%%%%%%%%%%%%%%%%
%定义装订线

\newsavebox{\zdx}

\newcommand{\putzdx}{\marginpar{
\parbox{1cm}{\vspace{-2.6cm}
\rotatebox[origin=c]{90}{
\usebox{\zdx}
}} }}

%%%%%%

\sbox{\zdx}
    {\parbox{27cm}{
    \centering \CJKfamily{song}{{}\vspace{13mm}(\  \rotatebox[origin=c]{270}{题}
    \rotatebox[origin=c]{270}{答}\
    \rotatebox[origin=c]{270}{要}\ \rotatebox[origin=c]{270}{不}
    \rotatebox[origin=c]{270}{内}\
    \rotatebox[origin=c]{270}{线}\ \rotatebox[origin=c]{270}{订}\ \rotatebox[origin=c]{270}{装}
\ )} \\
\vspace{2mm}
\hspace{-40mm}\dotfill{} \dotfill{}\dotfill{}\dotfill{}\hspace{-100mm}{} \\
}}

\reversemarginpar

%第一页
\putzdx %%装订线--奇页数

%% 试卷头

  \begin{center}
    \begin{Large}
       复旦大学数学科学学院\\
        2005~2006学年第二学期期末考试试卷\vspace{2mm}\\
        {} $\Box$  \textbf{A 卷} \hspace{1cm}$\Box$ \textbf{B 卷}
    \end{Large}
  \end{center}

\noindent\large
\textbf{课程名称}:\kcmc{数学分析II}\hspace{2mm}\textbf{课程代码}:\kcdm{318.151.2}\\
\textbf{开课院系}:\kcmc{数学科学学院}\hspace{2mm}\textbf{考试形式}:\ksxs{闭卷}\\
\textbf{姓\ \ 名}:\underline{\, \hspace{2.2cm}}\hspace{2mm}
\textbf{学\ \ 号}: \underline{\, \hspace{2.0cm}}\hspace{2mm}
\textbf{专\ \ 业}:\underline{\, \hspace{3.2cm}}
%在“数学分析II”的位置填入课程名称;在“318.151.2”的位置填入课程代码
%在“闭卷”的位置填入考试形式:开卷、闭卷或课程论文

\bigskip

\begin{table}[htbp]\centering
\begin{tabular}{|c|c|c|c|c|c|c|c|c|c|}
\hline
\textbf{题\ \ 目}& 1 & 2 & 3 & 4 & 5 & 6 & 7 & 8 & \textbf{总分}\\
\hline \textbf{得\ \ 分}& \hspace{10mm} & \hspace{10mm} &
\hspace{10mm} &\hspace{10mm}
& \hspace{10mm} & \hspace{10mm} & \hspace{10mm} & \hspace{10mm} & \hspace{10mm} \\
\hline
\end{tabular}
\end{table}

\bigskip

\bigskip
%如果题目多于八题

\noindent
%\begin{table}[htbp]\centering
\begin{tabular}{|c|c|c|c|c|c|c|c|c|c|}
\hline
\textbf{题\ \ 号}& 1 & 2 & 3 & 4 & 5 & 6 & 7 & 8 & \multirow{2}{10pt}{\textbf{总\ \ 分}}\\
\cline{1-9} \textbf{得\ \ 分}& \hspace{10mm} & \hspace{10mm} &
\hspace{10mm} &\hspace{10mm}
& \hspace{10mm} & \hspace{10mm} & \hspace{10mm} & \hspace{10mm} & \hspace{10mm} \\
\hline
\textbf{题\ \ 号} & 9 & 10 & 11 & 12 & 13 & 14 & 15 & 16 &  \multirow{2}{10pt}{} \\
\cline{1-9} \textbf{得\ \ 分} & \hspace{10mm} & \hspace{10mm} &
\hspace{10mm} &\hspace{10mm}
& \hspace{10mm} & \hspace{10mm} & \hspace{10mm} & \hspace{10mm}& \hspace{10mm}\\
\hline
\end{tabular}

\bigskip

\bigskip\noindent \textbf{填空题}\begin{enumerate}\item
已知函数~$f(\log_ax)=\sqrt{x}$,则~$f(x)=\uline{\hspace{1.5 cm}}$,其定义域为$\uline{\hspace{1.5 cm}}$,
其中~$a\neq 0$,且~$a>0$~。
\item
设~$f(x)=\frac{x}{\sqrt{1+x^2}}, f_1(x)=f[f(x)], f_2(x)=f[f_1(x)],\cdots, f_{n+1}(x)=f[f_n(x)](n=1, 2,\cdots)$.
则~$f_n(x)=\uline{\hspace{1.5 cm}}$~.
\end{enumerate}
\noindent \textbf{选择题}\begin{enumerate}\item
函数~$f(x)=\frac{1}{1+\frac{1}{1+\frac{1}{x}}}$~的定义域为$\uline{\hspace{1.5 cm}}$。
\begin{flalign*}
&(A)~x\in \mathbf{R}, \text{但} x\neq 0.& &(B)~x\in \mathbf{R}, \text{但} 1+\frac{1}{x}\neq 0. \\
&(C)~x\in \mathbf{R}, \text{但} x\neq 0, -1, -\frac{1}{2}.& &(D)~x\in \mathbf{R}, \text{但} x\neq 0, -1.
\end{flalign*}
\item
设
$
f(x)=
\begin{cases}
e^{-x}, &x\leq 0\\
\cos x, &x>0
\end{cases}
$
则~$f(-x)=\uline{\hspace{1.5 cm}}$~
\begin{flalign*}
&(A)~f(-x)=
\begin{cases}
-e^{-x}, &x\leq 0\\
-\cos x, &x>0
\end{cases}.
& &(B)f(-x)=
\begin{cases}
e^{-x}, &x\leq 0\\
\cos x, &x>0
\end{cases}. \\
&(C)~f(-x)=
\begin{cases}
-\cos x, &x<0\\
e^{-x}, &x\geq 0
\end{cases}.
& &(D)f(-x)=
\begin{cases}
\cos x, &x<0\\
e^x, &x\geq 0
\end{cases}.
\end{flalign*}
\item
设函数~$f(x)$~为奇函数,~$g(x)$~为偶函数,且它们可以构成复合函数\\
$$f[f(x)], g[f(x)], f[g(x)], g[g(x)],$$
则其中为奇函数的是~$\uline{\hspace{1.5 cm}}.$\\
(A)~$f[f(x)]$~\quad (B)~$g[f(x)]$~\quad (C)~$f[g(x)]$~\quad (D)~$g[g(x)]$~
\item
选择以下两题中给出的四个结论中一个正确的结论:\\
(1)设在~$[0, 1]$~上~$f^{\prime\prime}(x)>0$~,则~$f^\prime(0)$~,~$f^\prime(1)$~,~$f(1)-f(0)$~或~$f(0)-f(1)$~几个数
的大小顺序为(\hspace{1.5 cm})。
\begin{flalign*}
&(A)~f^\prime(1)>f^\prime(0)>f(1)-f(0).& &(B)~f^\prime(1)>f(1)-f(0)>f^\prime(0).~ \\
&(C)~f(1)-f(0)>f^\prime(1)>f^\prime(0).& &(D)~f^\prime(1)>f(0)-f(1)>f^\prime(0).~
\end{flalign*}
(2)设~$f^\prime(x_0)=f^{\prime\prime}(x_0)=0$~,$f^{\prime\prime\prime}(x_0)>0$,则(\hspace{1.5 cm})。\\
(A)~$f^\prime(x_0)$~是~$f^\prime(x)$~的极大值。\\
(B)~$f(x_0)$~是~$f(x)$~的极大值。\\
(C)~$f(x_0)$~是~$f(x)$~的极小值。\\
(D)~$(x_0, f(x_0))$~是曲线~$y=f(x)$的拐点。
\end{enumerate}
\noindent \textbf{判断题}\begin{enumerate}\item
设~$f(x)$~为奇函数,判断下列函数的奇偶性:\\
\begin{flalign*}
&~(1)~xf(x)~& &~(2)~(x^2+1)f(x)~& &~(3)~|f(x)|~\\
&~(4)~-f(-x)~& &~(5)~f(x)(\frac{2^x+1}{2}-\frac{1}{2})~& &\\
\end{flalign*}
\item
判断函数~$f(x)=\ln(x+\sqrt{1+x^2})$~的奇偶性。
\end{enumerate}
\clearpage
\end{CJK}
\end{document}